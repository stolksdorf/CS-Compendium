
\begin{tikzpicture}
	[scale=.6,auto=right,every node/.style={circle}]
	\node (n6) at (1,10) {6};
	\node (n4) at (4,8)  {4};
	\node (n5) at (8,9)  {5};
	\node (n1) at (11,8) {1};
	\node (n2) at (9,6)  {2};
	\node (n3) at (5,5)  {3};

	\foreach \from/\to in {n6/n4,n4/n5,n5/n1,n1/n2,n2/n5,n2/n3,n3/n4}
	\draw[->] (\from) -- (\to);
	\draw[->] (n6) .. controls +(-30:3cm) and +(-150:6cm) .. (n3);

\end{tikzpicture}

\subsection{Trees}

\Tree[.IP [.NP [.Det \textit{the} ]
				 [.N\1 [.N \textit{package} ]]]
			[.I\1 [.I \textsc{3sg.Pres} ]
				[.VP [.V\1 [.V \textit{is} ]
							 [.AP [.Deg \textit{really} ]
								[.A\1 [.A \textit{simple} ]
										\qroof{\textit{to use}}.CP ]]]]]]


\tikzstyle{every node}=[draw=black,thick,anchor=west]
\tikzstyle{selected}=[draw=red,fill=red!30]
\tikzstyle{optional}=[dashed,fill=gray!50]
\begin{tikzpicture}[%
	grow via three points={one child at (0.5,-0.7) and
	two children at (0.5,-0.7) and (0.5,-1.4)},
	edge from parent path={(\tikzparentnode.south) |- (\tikzchildnode.west)}]
	\node {texmf}
	child { node {doc}}
	child { node {fonts}}
	child { node {source}}
	child { node [selected] {tex}
		child { node {generic}}
		child { node [optional] {latex}}
		child { node {plain}}
	}
	child [missing] {}
	child [missing] {}
	child [missing] {}
	child { node {texdoc}};
\end{tikzpicture}





Visitor
Type: Behavioral
What it is: Represent an operation to be performed on the elements of an object structure. Lets you define a new operation without changing the classes of the elements on which it operates.

Template Method
Type: Behavioral
What it is: Define the skeleton of an algorithm in an operation, deferring some steps to subclasses. Lets subclasses redefine certain steps of an algorithm without changing the algorithm's structure.

Strategy
Type: Behavioral
What it is: Define a family of algorithms, encapsulate each one, and make them interchangeable. Lets the algorithm vary independently from clients that use it.

State
Type: Behavioral
What it is: Allow an object to alter its behavior when its internal state changes. The object will appear to change its class.

Subject Observer
Type: Behavioral
What it is: Define a one-to-many dependency between objects so that when one object changes state, all its dependents are notified and updated automatically.

Memento
Type: Behavioral
What it is: Without violating encapsulation, capture and externalize an object's internal state so that the object can be restored to this state later.

Chain of Responsibility
Type: Behavioral
What it is: Avoid coupling the sender of a request to its receiver by giving more than one object a chance to handle the request. Chain the receiving objects and pass the request along the chain until an object handles it.

Invoker Command
Type: Behavioral
What it is: Encapsulate a request as an object, thereby letting you parameterize clients with different requests, queue or log requests, and support undoable operations.

Interpreter
Type: Behavioral
What it is: Given a language, define a representation for its grammar along with an interpreter that uses the representation to interpret sentences in the language.

Iterator
Type: Behavioral
What it is: Provide a way to access the elements of an aggregate object sequentially without exposing its underlying representation.

Mediator
Type: Behavioral
What it is: Define an object that encapsulates how a set of objects interact. Promotes loose coupling by keeping objects from referring to each other explicitly and it lets you vary their interactions independently.


Adapter
Type: Structural
What it is:Convert the interface of a class intoanother interface clients expect. Letsclasses work together that couldn'totherwise because of incompatibleinterfaces.

Bridge
Type: Structural
What it is:Decouple an abstraction from itsimplementation so that the two can varyindependently.

Composite
Type: Structural
What it is:Compose objects into tree structures torepresent part-whole hierarchies. Letsclients treat individual objects andcompositions of objects uniformly.

Decorator
Type: Structural
What it is:Attach additional responsibilities to anobject dynamically. Provide a flexiblealternative to sub-classing for extendingfunctionality.

Facade
Type: Structural
What it is:Provide a unified interface to a set ofinterfaces in a subsystem. Defines a high-level interface that makes the subsystemeasier to use.

Flyweight
Type: Structural
What it is:Use sharing to support large numbers offine grained objects efficiently.

Proxy
Type: Structural
What it is:Provide a surrogate or placeholder foranother object to control access to it.

Singleton
Type: Creational
What it is:Ensure a class only has one instance andprovide a global point of access to it.

Prototype
Type: Creational
What it is:Specify the kinds of objects to createusing a prototypical instance, andcreate new objects by copying thisprototype.

Factory Method
Type: Creational
What it is:Define an interface for creating anobject, but let subclasses decide whichclass to instantiate. Lets a class deferinstantiation to subclasses.

Builder
Type: Creational
What it is:Separate the construction of acomplex object from its representingso that the same constructionprocess can create different representations.

Abstract Factory
Type: Creational
What it is:Provides an interface for creatingfamilies of related or dependentobjects without specifying their concrete class.



\fi

